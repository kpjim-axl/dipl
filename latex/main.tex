\documentclass[12px]{report}
%\usepackage[english, greek]{babel}
%\usepackage[utf8]{inputenc}
\usepackage{graphicx}
\usepackage{amsfonts}
\usepackage{xltxtra} 
\usepackage{xgreek} 
%\setmainfont[Mapping=tex-text]{GFS Didot} 
\setmainfont[Mapping=tex-text]{Times New Roman} 

\title{
	{Thesis Title}\\
	{\large National Technical Uni of Athens}\\
%	{\includegraphics{uni.jpg}}
}
\author{Κωνσταντίνος Παπαδημητρίου}
\date{Day Month Year}

\maketitle
\begin{document}
\chapter*{Abstractgr}
Το πρόβλημα διαμοιρασμού των πόρων ενός συστήματος στις διεργασίες που το
χρησιμοποιούν είναι από τα σημαντικότερα που καλείται να επιλύσει ένα
λειτουργικό σύστημα. Ανα τα χρόνια έχουν προταθεί πολλοί αλγόριθμοι και
διαφορετικές πολιτικές που προσπαθούν να διαμοιράσουν με τέτοιο τρόπο τους
πόρους στις διεργασίες με απώτερο στόχο τη βέλτιστη συνολική απόδοση του
συστήματος.
\par
Τα τελευταία χρόνια χρησιμοποιούνται σε βιομηχανικό (???) επίπεδο οι λεγόμενες εικονικές μηχανές. Με τη χρήση της
εικονοποίησης, επιλύονται προβλήματα που προηγουμένως αποτελούσαν μεγάλο εμπόδιο στην ανάπτυξη και στην παραγωγικότητα
μίας εταιρείας. Πλέον
\chapter*{Dedication}
Το κείμενο αφιερώνεται στους
\end{document}
