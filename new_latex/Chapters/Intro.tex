\chapter{Εισαγωγή}
\section{Αρχές πολυπύρηνων αρχιτεκτονικών}
Τα τελευταία χρόνια η σχεδίαση και οι αρχές των υπολογιστικών συστημάτων έχουν
αλλάξει ραγδαία. Αρχικά ένας επεξεργαστής αποτελούνταν από ένα μόλις πυρήνα ο
οποίος  αναλάμβανε να εκτελέσει όλη την εργασία. Αυτό φυσικά καυτστερεί
ιδιαίτερα τις διάφορες εφαρμογές και δεν εκμεταλεύεται τις δυνατότητες
παραλληλισμού είτε εντός της εφαρμογής είτε μεαταξύ δύο η περισσότερων
εφαρμογών. Έτσι η πρόοδος της τεχνολογίας και οι απαιτήσεις για μεγάλη
υπολογιστική δύναμη έφεραν τις πολυπύρηνες αρχιτεκτονικών, στις οποίες ο φόρτος
εργασίας διαμοιράζεται σε περισσότερους πυρήνες, βελτιώνοντας θεαματικά την
απόδοση ενός συστήματος.

Η τεχνολογία των πολυπύρηνων αρχιτεκτονικών ήταν γνωστή και χρησιμοποιούταν από
τις προηγούμενες δεκαετίες σε συστήματα μεγάλης κλίμακας όπως υπερυπολογιστές
και κέντρα δεδομένων (data centers). Είναι τα τελευταία χρόνια όμως που έχει
γνωρίσει μεγάλη άνθηση. Η μεγάλη ζήτηση σε αποδοτικότερα συστήματα έφερε την
τεχνολογία σε υπολογιστές και συστήματα γενικού σκοπού όπως laptop και κινητά
τηλέφωνα.
Ωστόσο, τα συστήματα αυτά συναντάνε νέα προβλήματα (ΠΡΕΠΕΙ ΝΑ ΒΑΛΩ ΤΑ ΠΡΟΒΛΗΜΑΤΑ
ΤΩΝ ΜΟΝΟΠΥΡΗΝΩΝ ΑΡΧΙΤΕΚΤΟΝΙΚΩΝ) τα οποία περιορίζουν την βελτίωση της
απόδοσης/υπολογιστικής ισχύος. Οι πυρήνες μοιράζονται κοινούς πόρους για τη
χρήση/χρησιμοποίηση των οποίων ανταγωνίζονται. Αυτό έχει ως αποτέλεσμα τον
περιορισμό της συνολικής επίδοσης (throughput). Είναι δουλειά λοιπόν του
λειτουργικού συστήματος να μοιράσει με τέτοιο τρόπο τους πόρους στις διάφορες
εφαρμογές ώστε να ελαχιστοποιείσει αυτόν τον ανταγωνισμό.
\section{Λειτουργικά Συστήματα}
Με τον όρο λειτουργικό σύστημα (Operating System - OS) εννοούμε το πρόγραμμα το
οποίο αναλαμβάνει να διαμοιράσει τους πόρους του συστήματος στις διάφορες
διεργασίες. Βασικός ρόλος λοιπόν του ΛΣ είναι να δίνει χρόνο για εκτέλεση στις
εφαρμογές και να διαμοιράζει την κύρια μνήμη όπως και τους λοιπούς πόρους.
Επίσης αναλαμβάνει να διατειρεί τα διάφορα συστήματα αρχείων και να παρέχει στις
διεργασίες ό,τι μπορεί να χρειαστούν.

Για την επίλυση του προαναφερθήσαντος προβλήματος το ΛΣ πρέπει να λαβμάνει υπ'
όψη τις απαιτήσεις των διεργασιών και να παίρνει διάφορες αποφάσεις αναφορικά με
τον διαμοιρασμό/την παροχή των πόρων. Βασικό συστατικό ενός ΛΣ αποτελεί ο
χρονοδρομολογητής (scheduler). Ο χρονοδρομολογητής παίρνει αποφάσεις βασισμένες
σε μια συγκεκριμένη πολιτική και αφορούν τη χρήση του επεξεργαστή από τις
διεργασίες με στόχο τη βέλτιστη απόδοση του συστήματος. Η απόδοση από τη μεριά
της μπορεί να μεταφράζεται είτε σε καλή αποκρισιμότητα - χαρακτηριστικό
απαραίτητο σε συστήματα γενικού σκοπού, είτε σε υψηλό throughput, δηλαδή
επεξεργασία μεγάλου όγκου δεδομένων ανα μονάδα χρόνου - απαραίτητο σε συστήματα
μεγάλης κλίμακας, όπως servers και υπερυπολογιστές.

Έχουν προταθεί και υλοποιηθεί λοιπόν πολλές πολιτικές/αλγόριθμοι που αποφασίζουν
την σειρά και με την οποία οι διάφορες διεργασίες θα καταλάβουν/χρησιμοποιήσουν
την κεντρική μονάδα επεξεργασίας (ΠΡΕΠΕΙ ΝΑ ΓΕΜΙΣΟΥΜΕ ΚΑΙ ΣΕΛΙΔΕΣ!!!). Για
συστήματα με μόνο μία μονάδα επεξεργασίας, το πρόβλημα ανάγεται στον διαμοιρασμό
του χρόνου στις διεργασίες που χρειάζονται τον επεξεργαστή και μετά από έρευνες
ετών οι πολιτικές που είχαν προταθεί βελτιστοποιήθηκαν. Με την πρόοδο
ωστόσο της τεχνολογίας και την είσοδο στις πολυπύρηνες αρχιτεκτονικές η
πολυπλοκότητα του προβλήματος αυξήθηκε καθώς πλέον δεν πρέπει να διαμοιραστεί
μόνο ο χρόνος αλλά να αποφασιστεί και ποιος πυρήνας θα διατεθεί στην εκάστοτε
διεργασία. Οι υπάρχουσες υλοποιήσεις που χρησιμοποιούνται στα σημερινά λειτουργικά
συστήματα δεν λαμβάνουν υπ' όψιν την πολυπλοκότητα του συστήματος και
χειρίζονται τους πυρήνες ως ξεχωριστές οντότητες. Αυτή η λογική ενώ είναι αρκετά
απλή και εύκολα υλοποιήσιμη, έχει συχνά ως συνέπεια την ταυτόχρονη εκτέλεση
διεργασιών που χρησιμοποιούν κοινούς πόρους, με αποτέλεσμα την αύξηση του
ανταγωνισμού και τον περιορισμό της απόδοσης του συστήματος.

..
\section{Υπολογιστικό Νέφος (Cloud Computing)}
Τα τελευταία χρόνια προωθείται με νέα τάση, όπου ο χρήστης δεν χρησιμοποιεί
μηχανήματα της κατοχής του για υπολογισμούς ή αποθήκευση δεδομένων, αλλά προτιμά
οργανισμούς ή εταιρείες που είναι σε θέση να παρέχουν τις εν λόγω υπηρεσίες.
Υπολογιστικό νέφος (Cloud Computing) είναι ένας τύπος υπηρεσίας, βασισμένης στο
διαδίκτυο, όπου ένας χρήστης μπορεί να χρησιμοποιήσει απομακρυσμένα συστήματα
για προσωπική χρήση. Έτσι μία οντότητα που μπορεί να είναι είτε απλός χρήστης είτε
οργανισμός ή εταιρεία έχει τη δυνατότητα να μισθώσει μηχανήματα που δεν έχει
στην κατοχή της και να γλιτώσει τουλάχιστον αρχικά το κόστος αγοράς και
εγκατάστης δικού της κέντρου δεδομένων. 

\subsection{Μοντέλα Νεφών}
Τα βασικά μοντέλα υπηρεσιών που παρέχει ένα νέφος χωρίζονται σε αφαιρετικά
επίπεδα και είναι τα εξής:

\begin{wrapfigure}{r}{0.38\textwidth}
	\centering
	\includegraphics[scale=0.5]{cloud}
\end{wrapfigure}

\textbf{Λογισμικό ως Υπηρεσία - Software as a Service (SaaS)}. Σε αυτόν τον
τύπο, ο τελικός χρήστης / πελάτης έχει την δυνατότητα να χρησιμοποιήσει τις
εφαρμογές που του δίνει ο πάροχος και τρέχουν στο νέφος. Οι εφαρμογές γίνονται
προσβάσιμες από τον χρήστη μέσω κάποιας διεπαφής (πχ web browser, command-line
interface κλπ). Ο πελάτης δεν έχει την δυνατότητα να τροποποιήσει τις υποδομές
του νέφους (όπως το δίκτυο, τους αποθηκευτικούς χώρους ή το λειτουργικό σύστημα)
παρα μόνο την εφαρμογή που του παρέχεται, τις ρυθμίσεις της και το περιβάλλον
της. Τυπικά παραδείγματα SaaS είναι εφαρμογές email, εικονικές επιφάνειες
εργασίας και online παιχνίδια.

\textbf{Πλατφορμα ως Υπηρεσία - Platform as a Service (PaaS)}. %to be filled
To be filed

\textbf{Υποδομή ως Υπηρεσία - Infrastructure as a Service (IaaS)}. Εδώ ο
χρήστης έχει τη δυνατότητα να δημιουργήσει και να ελέγξει το σύστημα που θα
χρησιμοποιήσει, δηλαδή να επιλέξει χαρακτηριστικά όπως το λειτουργικό σύστημα η
τοπολογία του δικτύου κ.ο.κ αλλά και πάλι δεν μπορεί να τροποποιήσει το
υποκείμενο σύστημα πάνω στο οποίο τρέχουν οι υπηρεσίες εκτός ίσως από
περιορισμένο αριθμό ρυθμίσεων, όπως το τοίχος προστασίας του host (firewall).
Τυπικό παράδειγμα αυτού του τύπου των υπηρεσιών είναι η εικονικές μηχανές και ο
εικονικός/απομακρυσμένος (?) αποθηκευτικός χώρος. Στην περίπτωση των εικονικών
μηχανών, ο χρήστης δημιουργεί εικονικά μηχανήματα, τα οποία υλοποιούνται με
κάποιον hypervisor στο απομακρυσμένο σύστημα και ο πελάτης έχει πλήρη πρόσβαση
σε αυτά. Ο τρόπος με τον οποίο τα εικονικά μηχανήματα τοποθετούνται στους
επεξεργαστές του host είναι το αντικείμενο μελέτης της παρούσας εργασίας.

\subsection{Πλεονεκτήματα}
Ίσως το σημαντικότερο πλεονέκτημα του νέφους είναι η σημαντική εξοικονόμηση κόστους
και ενέργειας. Ο χρήστης έχει τις υπηρεσίες που χρειάζεται, όταν τις χρειάζεται
χωρίς να απασχολείται για το που και πως θα βρει την υπολογιστική ισχύ που
απαιτείται και χωρίς να πρέπει να αγοράσει υπερκοστολογιμένα μηχανήματα.
Αντίστοιχα μία εταιρεία μπορεί να είναι πλήρως λειτουργική με σχεδόν μηδενικό
κόστος.

Επίσης πολύ σημαντική είναι η ευκολία και η διαθεσιμότητα των υπηρεσιών από
οποιοδήποτε σημείο βρίσκεται ο τελικός χρήστης. Με απλές μεθόδους (login) ο
χρήστης έχει πλήρη πρόσβαση στις εικονικές μηχανές του ή στα απομακρυσμένα
δεδομένα που έχει αποθηκεύσει. Άλλωστε ο πάροχος αναλαμβάνει την προστασία των
δεδομένων και την ασφαλή επαναφορά τους μετά από απώλεια, κρατώντας πολλά αντίγραφα
στο νέφος. Σε περίπτωση κάποιας αναπόφευκτης αστοχίας, ο τελικός χρήστης δεν
θα χρειαστεί να διαθέσει ούτε χρόνο ούτε ανθρώπινο δυναμικό για την επισκευή
βλαβών.

Η χρήση του νέφους παρέχει μεγάλη ευελιξία. Το κόστος (είτε χρηματικό, είτε
ενεργειακό, είτε εξοπλιστικό) είναι άμεσα συνδεδεμένο με τις !current! ανάγκες
του χρήστη. Έτσι αν μία εταιρεία χρειαστεί παραπάνω πόρους για συγκεκριμένο /
προκαθορισμένο χρονικό διάστημα το μόνο που έχει να κάνει είναι να τους
νοικιάσει από κάποιο πάροχο αντί να υποστεί το πλήρες αντίτιμο για αυτούς (ή
κάτι τέτοιο τεσπα)

\subsection{Μειονεκτήματα}
Όπως όλες οι νέες τεχνολογίες, εκτός από θετικά υπάρχουν και κάποια αρνητικά. Με
τη χρήση του νέφους, υπάρχει ο -έστω και αμυδρός- κίνδυνος τεχνικής βλάβης για
την οποία δεν ευθύνεται ο χρήστης. Τυπικά οι πόροι είναι μόνιμα διαθέσιμοι
από διαφορετικές τοποθεσίες, ωστόσο η πρόσβαση σε αυτούς γίνεται μέσω
διαδικτύου που ακόμα και σήμερα δεν θεωρείται πάντα δεδομένη.

Επίσης σημαντική είναι η ασφάλεια των δεδομένων. Μία εταιρεία εμπιστεύεται μέρος
των δεδομένων της σε τρίτους, χωρίς να μπορεί να διασφαλίσει πως αυτά δεν θα
υποκλαπούν. Άλλωστε η κρυπτογράφησή τους μπορεί να μειώσει πολύ την απόδοση του
εγχειρήματος. Παρόμοια, η ιδιοτικότητα ευαίσθητων δεδομένων δεν μπορεί να
θεωρείται δεδομένη.

\section{Εικονικές Μηχανές (VM)}
Ίσως παρακάτω να βάλεις και εναλλακτικές (containers κλπ) και να κάνεις μία
μικρή σύγκριση...
Η έννοια των εικονικών μηχανών υπάρχει από την δεκαετία του 1960 αλλά είναι τα
τελευταία 10 χρόνια που έχει επανέλθει στο προσκήνιο, κυρίως επειδή η υψηλή
υπολογιστική ισχύς των σύγχρονων συστημάτων προσφέρει τη δυνατότητα υλοποίησής
τους. Με τον όρο εικονοποίηση, εννοούμε τη δημιουργία εικονικών αντικειμένων
(αντί πραγματικών) όπως εικονικές πλατφόρμες, δίκτυα, συσκευές αποθήκευσης κλπ.
Τα πλεονεκτήματα από τη χρήση εικονικών μηχανών είναι πολλά. Κάποια είναι
περισσότερο  προφανή, όπως η εκτέλεση εφαρμογών που δεν έχουν σχεδιαστεί για
συγκεκριμένο υπολογιστικό/λειτουργικό σύστημα και η χρήση συσκευών που δεν είναι
μέρος ενός συστήματος, αλλά υλοποιημένες σε λογισμικό, ενώ άλλα είναι λιγότερο
προφανή και έχουν να κάνουν με την ασφάλεια και την απομόνωση που παρέχει μία
εικονική μηχανή.
Υπάρχουν τρεις βασικοί/ές τύποι/τεχνικές εικονοποίησης: \textbf{Full
virtualization}, \textbf{Partial Virtualization} και \textbf{Paravirtualization}.

Ο/Η πρώτος/η τύπος/τεχνική σηματοδοτεί την πλήρη προσομοίωση του υλικού σε
λογισμικό. Αυτό σημαίνει πως το λειτουργικό σύστημα του guest (VM) δεν γνωρίζει
πως τρέχει σε προσομοίωση, δηλαδή δεν έχει υποστεί καμία αλλαγή. Αυτού του τύπου
η εικονοποίηση παρέχει το πλεονέκτημα πως το λειτουργικό δεν γνωρίζει πως τρέχει
σε εικονική μηχανή και συμπεριφέρεται ακριβώς σαν να έτρεχε σε πραγματικό
σύστημα. Ωστόσο, αφού όλο το υλικό είναι υλοποιημένο σε λογισμικό, η όλη
διαδικασία υστερεί σε απόδοση.

Με την δεύτερη τεχνική η εικονική μηχανή προσομοιώνει επαρκές τμήμα του
πραγματικού υποκείμενου υλικού ώστε να επιτρέπει την εκτέλεση ενός μη
τροποποιημένου ΛΣ σχεδιασμένου για την ίδια αρχιτεκτονική επεξεργαστή με αυτήν
του πραγματικού. Σε αυτήν την περίπτωση δεν χρειάζεται η πλήρης εξομοίωση του
συνόλου εντολών του πραγματικού επεξεργαστή και μάλιστα υπό συνθήκες επιτρέπεται
απευθείας εκτέλεση των εντολών του φιλοξενούμενου μηχανήματος στο πραγματικό με
την προϋπόθεση να μην επιρρεάζεται κάποιο υποσύστημα έξω από τον άμεσο έλεγχό
του. Σε κρίσιμα τμήματα ωστόσο (πχ σε προσπάθεια πρόσβασης σε συσκευή μέσω κλήσης
συστήματος) η εποπτεία του hypervisor είναι αναπόφευκτη.

Με την τρίτη τεχνική, το λειτουργικό σύστημα του guest γνωρίζει την ύπαρξη του
hypervisor (host) και συνεργάζεται με αυτόν για την καλύτερη απόδοση του
συστήματος. Ο hypervisor σε αυτήν την περίπτωση προσφέρει μία διεπαφή στο guest
os που του επιτρέπει την (απ)ευθεία(ς) χρήση του υλικού. Για την χρήση ωστόσο
αυτής της διεπαφής το guest os πρέπει να υποστεί βασικές αλλαγές. Για να γίνει
χρήση αυτής της τεχνικής με ικανοποιητική βελτίωση της απόδοσης, απαιτείται
πρόσθετη υποστήριξη από το υλικό (Intel VT-x, AMD-V).

Τα πλεονεκτήματα της χρήσης εικονικών μηχανών είναι πολλά και όπως προαναφέρθηκε
κάποια είναι περισσότερο προφανή από άλλα:
\begin{itemize}
	\item Δημιουργία υλικών συσκευών από λογισμικό. Με χρήση εικονικών μηχανών,
		ένας προγραμματιστής είναι σε θέση να δημιουργήσει συσκευές χωρίς την
		αναγκαστική αγορά ακριβού υλικού.
	\item Εκτέλεση εφαρμογών σχεδιασμένων για συστήματα διαφορετικά από αυτό
		του host. Είναι απλή η προσομοίωση διάφορων αρχιτεκτονικών με
		συνέπεια ο χρήστης να έχει τη δυνατότητα να χρησιμοποιήσει
		λογισμικό που δεν είναι	σχεδιασμένο για τον host. Με την ίδια
		λογική μπορεί ο χρήστης να εκτελέσει εφαρμογές γραμμένες για ένα
		λειτουργικό σύστημα σε κάποιο άλλο (πχ παιχνίδια γραμμένα για
		Windows μπορούν  να τρέξουν σε host με Linux).
	\item Ασφάλεια του host μηχανήματος. Ο χρήστης έχει πλήρη πρόσβαση
		(administrator/root) στο δικό του εικονικό μηχάνημα, ωστόσο στον host
		διατηρεί δικαιώματα απλού χρήστη.
	\item Απομόνωση των εικονικών μηχανών. Ένα εικονικό μηχάνημα μπορεί να
		τρέχει στο ίδιο φυσικό μηχάνημα μαζί με πολλά άλλα. Ωστόσο ο χώρος του
		κάθε ενός είναι ιδιοτικός και δεν επιρρεάζεται από κάποιο άλλο μηχάνημα.
	\item Αποθήκευση μίας κατάστασης και επαναφορά σε αυτήν (πχ ύστερα από
		κάποια αστοχία (failover)) με χρήση στιγμιοτύπων (snapshot). Η χρήση
		στιγμιοτύπων επεξηγήται αναλυτικότερα στη συνέχεια.
	\item Χρήση νέων μηχανημάτων δίχως την αγορά νέου υλικού.
	\item Πειραματισμός και χρήση στην εκπαίδευση Λειτουργικών Συστημάτων
		χωρίς τον φόβο της καταστροφής ενός συστήματος από
		άπειρους/απείραστους χρήστες. Όπως αναφέρθηκε προηγουμένως, η χρήση
		εικονικών μηχανών προσφέρει ασφάλεια στον host. Όταν λοιπόν θέλουμε να
		αλλάξουμε το λειτουργικό σύστημα και να το προσαρμόσουμε στις ανάγκες
		μας, στο χειρότερο σενάριο θα επηρεαστεί μόνο το εικονικό μηχάνημα και
		όχι ο host.
\end{itemize}

\subsection{Στιγμιότυπο}
Όταν ο χρήστης θέλει να αποθηκεύσει την κατάσταση της εικονικής μηχανής του,
αποθηκεύει ένα \textbf{στιγμιότυπο (snapshot)}. Με το στιγμιότυπο το σύνολο των
δεδομένων (τόσο από τους σκληρούς δίσκους όσο και από την μνήμη RAM) του
εικονικού μηχανήματος αποθηκεύονται και μπορούν σε μελλοντικό χρόνο να
επαναφερθούν και η χρήση του μηχανήματος να συνεχίσει από εκεί. Αυτό είναι
και από τα χρησιμότερα χαρακτηριστικά μίας εικονικής μηχανής καθώς καθιστά την
επαναφορά της σε παρελθοντική κατάσταση ιδιαίτερα εύκολη (πχ μετά από κάποια
αστοχία).

\subsection{Migration}
Με τον όρο migration αναφερόμαστε στη μετακίνηση μίας εικονικής μηχανής από ένα
φυσικό μηχάνημα σε κάποιο άλλο. Αυτό γίνεται είτε για λόγους ανακατανομής του
φόρτου εργασίας είτε για τη συντήρηση του συστήματος είτε λόγω κάποιας αστοχίας
του φυσικού μηχανήματος.

Ενώ σε κάποιες περιπτώσεις η μετακίνηση αυτή είναι
απαραίτητη (συντήρηση - αστοχίες) σε άλλες είναι αρκετά απλό να αποφευχθεί.
Κατα τη μετακίνηση μίας εικονικής μηχανής η κατάσταση της μνήμης αντιγράφεται
από το ένα φυσικό μηχάνημα στο άλλο (είτε on the fly είτε μετά από αναστολή της
λειτουργίας της). Η διαδικασία αυτή είναι ιδιαίτερα απαιτητική σε χρόνο και
ενέργεια, οπότε είναι ιδιαίτερα σημαντικό να γίνεται όσο το δυνατό πιο σπάνια.

Γενικά για να είναι αποδοτικό ένα σύστημα, πρέπει οι πόροι του να βρίσκονται σε
χρήση όσο το δυνατό περισσότερο. Μετά από κάποια ώρα όμως είναι πολύ πιθανό να
υπάρξει κάποια ανισορροπία στο φόρτο των επεξεργαστών, δηλαδή ενώ κάποιος δεν θα
εκτελεί μία εργασία ένας άλλος θα είναι ιδιαίτερα επιβαρυμένος. Σε αυτή την
περίπτωση γίνεται ανακατανομή του φόρτου και η εκτέλεση των εργασιών μοιράζεται
εκ νέου στα διαθέσιμα μηχανήματα. Είναι ευθύνη του λειτουργικού συστήματος και
του χρονοδρομολογητή να κάνει αυτές τις ανισορροπίες και τις επακόλουθες
ανακατανομές όσο το δυνατό λιγότερες. Έτσι με μια πολιτική δρομολόγησης μπορεί
να έχουμε πολύ συχνές μετακινήσεις εικονικών μηχανών από το ένα μηχάνημα στο
άλλο, ενώ με μία καταλληλότερη οι μετακινήσεις αυτές να μειωθούν στο ελάχιστο.

\section{Ορισμός του προβλήματος}
Είναι κατανοητό πως όταν πολλές διεργασίες μοιράζονται κοινούς πόρους συνολική
απόδοση του συστήματος ως ένα βαθμό περιορίζεται. Έχουν προταθεί λοιπόν διάφορες
πολιτικές δρομολόγησης ώστε να περιοριστεί ο ανταγωνισμός στο ελάχιστο. Όταν
ένας χρήστης νοικιάζει/μισθώνει εικονικές μηχανές από/σε έναν πάροχο (συνήθως)
πληρώνει με την ώρα που βρίσκονται οι εικονικές του μηχανές πάνω στη cpu. Κάτι
τέτοιο σημαίνει πως είναι σημαντικό αυτή η ώρα να είναι η ελάχιστη δυνατή. Είναι
ταυτόχρονα προφανές πως όσο μικρότερος είναι ο αναγωνισμός, τόσο καλύτερη είναι
και χρήση των πόρων ενός συστήματος με αποτέλεσμα την ελαχιστοποίηση τόσο του
κόστους λειτουργίας, όσο και της κατανάλωσης ενέργειας.

\subsection{Προηγούμενη δουλειά}
Άλλοι έχουν προσεγγίσει το πρόβλημα ως εξής: μπλα μπλα μπλα μπλα

\subsection{Η δική μας προσέγγιση}
Γνωρίζοντας εκ των προτέρων λεπτομέρειες της αρχιτεκτονικής του συστήματος, όπως
την ιεραρχεία της μνήμης και τους κοινούς πόρους είμαστε σε θέση να προβλέψουμε
τον ανταγωνισμό και να υπολογίσουμε την επίδραση που θα έχει αυτός στην εκτέλεση
των εφαρμογών. Σε αυτή την εργασία, μελετούμε την παράλληλη εκτέλεση εικονικών
μηχανών και προσπαθούμε να αναλύσουμε την επίδραση που έχει ο ανταγωνισμός για
τους κοινούς πόρους στην φιλοξενεία τους σε ένα σύστημα. Για τον σκοπό αυτό
χρησιμοποιούμε προγράμματα ελέγχου (benchmarks) τα οποία εκτελούνται σε
περιβάλλον εικονικων μηχανών - εικονοποίησης. Τα προγράμματα αυτά υλοποιούν
διάφορους αλγορίθμους, κυρίως γραμμικής άλγεβρας και το καθένα έχει διαφορετικές
απαιτήσεις τόσο σε μνήμη όσο και στον δίαυλο προς αυτή. Αφού τα διαχωρίσουμε με
βάση την ανάγκες τους σε διαφορετικές κλάσεις (classification), κάνουμε
πειράματα με όλους τις πιθανούς συνδιασμούς κλάσεων ώστε να δούμε με ποια
πολιτική δρομολόγησης έχουμε την καλύτερη επίδοση. Ιδιαίτερο ενδιαφέρον
παρουσιάζει και το κατα πόσο το εικονικό περιβάλλον στο οποίο εκτελλούνται
επηρεάζει την απόδοσή τους.

Για τους σκοπούς της εργασίας, χρησιμοποιούμε διάφορους χρονοδρομολογητές όπως ο
cfs του linux καθώς και πολιτικές που προσπαθούν να εξισορροπήσουν διάφορους
παράγοντες που επιρρεάζουν την επίδοση όπως τα cache misses. Τέλος δίνουμε
ιδιαίτερη βάση στο χρονοδρομολογητή lca που προτείνουμε.

\section{Περιγραφή κεφαλαίων}
Στο δεύτερο κεφάλαιο αναλύουμε σε μεγαλύτερο βάθος το πρόβλημα διατυπώνοντας
τους παράγοντες που το αποτελούν. Επίσης αναφέρουμε σχετική έρευνα από τη
βιβλιογραφία και αναλύουμε έναν σχετικά απλό αλλά αποδοτικό αλγόριθμο από τον
οποίο αναμένουμε καλά αποτελέσματα.

Στο τρίτο κεφάλαιο περιγράφουμε το σύστημα και τις συνθήκες κάτω από τις οποίες
εκτελέστηκαν τα πειράματα, ενώ στο τέταρτο παρουσιάζουμε τα αποτελέσματα. Τέλος
το πέμπτο κεφάλαιο περιέχει μία ανακεφαλαίωση και τα συμπεράσματα της εργασίας
καθώς και μελλοντική δουλειά που μπορεί να γίνει.
