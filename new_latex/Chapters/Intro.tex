\chapter{Εισαγωγή}
\section{Αρχές πολυπύρηνων αρχιτεκτονικών}
Τα τελευταία χρόνια η σχεδίαση και οι αρχές των υπολογιστικών συστημάτων έχουν
αλλάξει ραγδαία. Αρχικά ένας επεξεργαστής αποτελούνταν από ένα μόλις πυρήνα ο
οποίος  αναλάμβανε να εκτελέσει όλη την εργασία. Αυτό φυσικά καυτστερεί
ιδιαίτερα τις διάφορες εφαρμογές και δεν εκμεταλεύεται τις δυνατότητες
παραλληλισμού είτε εντός της εφαρμογής είτε μεαταξύ δύο η περισσότερων
εφαρμογών. Έτσι η πρόοδος της τεχνολογίας και οι απαιτήσεις για μεγάλη
υπολογιστική δύναμη έφεραν τις πολυπύρηνες αρχιτεκτονικών, στις οποίες ο φόρτος
εργασίας διαμοιράζεται σε περισσότερους πυρήνες, βελτιώνοντας θεαματικά την
απόδοση ενός συστήματος.

Η τεχνολογία των πολυπύρηνων αρχιτεκτονικών ήταν γνωστή και χρησιμοποιούταν από
τις προηγούμενες δεκαετίες σε συστήματα μεγάλης κλίμακας όπως υπερυπολογιστές
και κέντρα δεδομένων (data centers). Είναι τα τελευταία χρόνια όμως που έχει
γνωρίσει μεγάλη άνθηση. Η μεγάλη ζήτηση σε αποδοτικότερα συστήματα έφερε την
τεχνολογία σε υπολογιστές και συστήματα γενικού σκοπού όπως laptop και κινητά
τηλέφωνα.
Ωστόσο, τα συστήματα αυτά συναντάνε νέα προβλήματα (ΠΡΕΠΕΙ ΝΑ ΒΑΛΩ ΤΑ ΠΡΟΒΛΗΜΑΤΑ
ΤΩΝ ΜΟΝΟΠΥΡΗΝΩΝ ΑΡΧΙΤΕΚΤΟΝΙΚΩΝ) τα οποία περιορίζουν την βελτίωση της
απόδοσης/υπολογιστικής ισχύος. Οι πυρήνες μοιράζονται κοινούς πόρους για τη
χρήση/χρησιμοποίηση των οποίων ανταγωνίζονται. Αυτό έχει ως αποτέλεσμα τον
περιορισμό της συνολικής επίδοσης (throughput). Είναι δουλειά λοιπόν του
λειτουργικού συστήματος να μοιράσει με τέτοιο τρόπο τους πόρους στις διάφορες
εφαρμογές ώστε να ελαχιστοποιείσει αυτόν τον ανταγωνισμό.
\section{Λειτουργικά Συστήματα}
Με τον όρο λειτουργικό σύστημα (Operating System - OS) εννοούμε το πρόγραμμα το
οποίο αναλαμβάνει να διαμοιράσει τους πόρους του συστήματος στις διάφορες
διεργασίες. Βασικός ρόλος λοιπόν του ΛΣ είναι να δίνει χρόνο για εκτέλεση στις
εφαρμογές και να διαμοιράζει την κύρια μνήμη όπως και τους λοιπούς πόρους.
Επίσης αναλαμβάνει να διατειρεί τα διάφορα συστήματα αρχείων και να παρέχει στις
διεργασίες ό,τι μπορεί να χρειαστούν.

Για την επίλυση του προαναφερθήσαντος προβλήματος το ΛΣ πρέπει να λαβμάνει υπ'
όψη τις απαιτήσεις των διεργασιών και να παίρνει διάφορες αποφάσεις αναφορικά με
τον διαμοιρασμό/την παροχή των πόρων. Βασικό συστατικό ενός ΛΣ αποτελεί ο
χρονοδρομολογητής (scheduler). Ο χρονοδρομολογητής παίρνει αποφάσεις βασισμένες
σε μια συγκεκριμένη πολιτική και αφορούν τη χρήση του επεξεργαστή από τις
διεργασίες με στόχο τη βέλτιστη απόδοση του συστήματος. Η απόδοση από τη μεριά
της μπορεί να μεταφράζεται είτε σε καλή αποκρισιμότητα - χαρακτηριστικό
απαραίτητο σε συστήματα γενικού σκοπού, είτε σε υψηλό throughput, δηλαδή
επεξεργασία μεγάλου όγκου δεδομένων ανα μονάδα χρόνου - απαραίτητο σε συστήματα
μεγάλης κλίμακας, όπως servers και υπερυπολογιστές.

Έχουν προταθεί και υλοποιηθεί λοιπόν πολλές πολιτικές/αλγόριθμοι που αποφασίζουν
την σειρά και με την οποία οι διάφορες διεργασίες θα καταλάβουν/χρησιμοποιήσουν
την κεντρική μονάδα επεξεργασίας (ΠΡΕΠΕΙ ΝΑ ΓΕΜΙΣΟΥΜΕ ΚΑΙ ΣΕΛΙΔΕΣ!!!). Για
συστήματα με μόνο μία μονάδα επεξεργασίας, το πρόβλημα ανάγεται στον διαμοιρασμό
του χρόνου στις διεργασίες που χρειάζονται τον επεξεργαστή και μετά από έρευνες
ετών οι πολιτικές που είχαν προταθεί βελτιστοποιήθηκαν. Με την πρόοδο
ωστόσο της τεχνολογίας και την είσοδο στις πολυπύρηνες αρχιτεκτονικές η
πολυπλοκότητα του προβλήματος αυξήθηκε καθώς πλέον δεν πρέπει να διαμοιραστεί
μόνο ο χρόνος αλλά να αποφασιστεί και ποιος πυρήνας θα διατεθεί στην εκάστοτε
διεργασία. Οι υπάρχουσες υλοποιήσεις που χρησιμοποιούνται στα σημερινά λειτουργικά
συστήματα δεν λαμβάνουν υπ' όψιν την πολυπλοκότητα του συστήματος και
χειρίζονται τους πυρήνες ως ξεχωριστές οντότητες. Αυτή η λογική ενώ είναι αρκετά
απλή και εύκολα υλοποιήσιμη, έχει συχνά ως συνέπεια την ταυτόχρονη εκτέλεση
διεργασιών που χρησιμοποιούν κοινούς πόρους, με αποτέλεσμα την αύξηση του
ανταγωνισμού και τον περιορισμό της απόδοσης του συστήματος.

..
\section{Υπολογιστικό Νέφος (Cloud Computing)}
Τα τελευταία χρόνια προωθείται με νέα τάση, όπου ο χρήστης δεν χρησιμοποιεί
μηχανήματα της κατοχής του για υπολογισμούς ή αποθήκευση δεδομένων, αλλά προτιμά
οργανισμούς ή εταιρείες που είναι σε θέση να παρέχουν τις εν λόγω υπηρεσίες.
Υπολογιστικό νέφος (Cloud Computing) είναι ένας τύπος υπηρεσίας, βασισμένης στο
διαδίκτυο, όπου ένας χρήστης μπορεί να χρησιμοποιήσει απομακρυσμένα συστήματα
για προσωπική χρήση. Έτσι μία οντότητα που μπορεί να είναι είτε απλός χρήστης είτε
οργανισμός ή εταιρεία έχει τη δυνατότητα να μισθώσει μηχανήματα που δεν έχει
στην κατοχή της και να γλιτώσει τουλάχιστον αρχικά το κόστος αγοράς και
εγκατάστης δικού της κέντρου δεδομένων. 

\subsection{Μοντέλα Νεφών}
Τα βασικά μοντέλα υπηρεσιών που παρέχει ένα νέφος χωρίζονται σε αφαιρετικά
επίπεδα και είναι τα εξής:

\begin{wrapfigure}{r}{0.38\textwidth}
	\centering
	\includegraphics[scale=0.5]{cloud}
\end{wrapfigure}

\textbf{Λογισμικό ως Υπηρεσία - Software as a Service (SaaS)}. Σε αυτόν τον
τύπο, ο τελικός χρήστης / πελάτης έχει την δυνατότητα να χρησιμοποιήσει τις
εφαρμογές που του δίνει ο πάροχος και τρέχουν στο νέφος. Οι εφαρμογές γίνονται
προσβάσιμες από τον χρήστη μέσω κάποιας διεπαφής (πχ web browser, command-line
interface κλπ). Ο πελάτης δεν έχει την δυνατότητα να τροποποιήσει τις υποδομές
του νέφους (όπως το δίκτυο, τους αποθηκευτικούς χώρους ή το λειτουργικό σύστημα)
παρα μόνο την εφαρμογή που του παρέχεται, τις ρυθμίσεις της και το περιβάλλον
της. Τυπικά παραδείγματα SaaS είναι εφαρμογές email, εικονικές επιφάνειες
εργασίας και online παιχνίδια.

\textbf{Πλατφορμα ως Υπηρεσία - Platform as a Service (PaaS)}. %to be filled
To be filed

\textbf{Υποδομή ως Υπηρεσία - Infrastructure as a Service (IaaS)}. Εδώ ο
χρήστης έχει τη δυνατότητα να δημιουργήσει και να ελέγξει το σύστημα που θα
χρησιμοποιήσει, δηλαδή να επιλέξει χαρακτηριστικά όπως το λειτουργικό σύστημα η
τοπολογία του δικτύου κ.ο.κ αλλά και πάλι δεν μπορεί να τροποποιήσει το
υποκείμενο σύστημα πάνω στο οποίο τρέχουν οι υπηρεσίες εκτός ίσως από
περιορισμένο αριθμό ρυθμίσεων, όπως το τοίχος προστασίας του host (firewall).
Τυπικό παράδειγμα αυτού του τύπου των υπηρεσιών είναι η εικονικές μηχανές και ο
εικονικός/απομακρυσμένος (?) αποθηκευτικός χώρος. Στην περίπτωση των εικονικών
μηχανών, ο χρήστης δημιουργεί εικονικά μηχανήματα, τα οποία υλοποιούνται με
κάποιον hypervisor στο απομακρυσμένο σύστημα και ο πελάτης έχει πλήρη πρόσβαση
σε αυτά. Ο τρόπος με τον οποίο τα εικονικά μηχανήματα τοποθετούνται στους
επεξεργαστές του host είναι το αντικείμενο μελέτης της παρούσας εργασίας.

\subsection{Πλεονεκτήματα}
Ίσως το βασικότερο πλεονέκτημα του νέφους είναι η σημαντική εξοικονόμηση κόστους
και ενέργειας. Ο χρήστης έχει τις υπηρεσίες που χρειάζεται, όταν τις χρειάζεται
χωρίς να απασχολείται για το που και πως θα βρει την υπολογιστική ισχύ που
απαιτείται και χωρίς να πρέπει να αγοράσει υπερκοστολογιμένα μηχανήματα.
Αντίστοιχα μία εταιρεία μπορεί να είναι πλήρως λειτουργική με σχεδόν μηδενικό
κόστος.

Επίσης πολύ σημαντική είναι η ευκολία και η διαθεσιμότητα των υπηρεσιών από
οποιοδήποτε σημείο βρίσκεται ο τελικός χρήστης. Με απλές μεθόδους (login) ο
χρήστης έχει πλήρη πρόσβαση στις εικονικές μηχανές του ή στα απομακρυσμένα
δεδομένα που έχει αποθηκεύσει. Άλλωστε ο πάροχος αναλαμβάνει την προστασία των
δεδομένων και την ασφαλή επαναφορά τους μετά από απώλεια, κρατώντας πολλά αντίγραφα
στο νέφος.

\subsection{Μειονεκτήματα}
Το βασικό μειονέκτημα του νέφους είναι ότι...

\section{Εικονικές Μηχανές (VM)}
Η έννοια των εικονικών μηχανών υπάρχει από την δεκαετία του 1960 αλλά είναι τα
τελευταία 10 χρόνια που έχει επανέλθει στο προσκήνιο, κυρίως επειδή η υψηλή
υπολογιστική ισχύς των σύγχρονων συστημάτων προσφέρει τη δυνατότητα υλοποίησής
τους.
