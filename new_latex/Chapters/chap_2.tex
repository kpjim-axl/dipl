\chapter{Τιτλος κεφαλαίου 2}
\section{Αρχες και λοιπά}
Στο δεύτερο κεφάλαιο αναλύουμε σε μεγαλύτερο βάθος το πρόβλημα διατυπώνοντας
τους παράγοντες που το αποτελούν. Επίσης αναφέρουμε σχετική έρευνα από τη
βιβλιογραφία και αναλύουμε έναν σχετικά απλό αλλά αποδοτικό αλγόριθμο από τον
οποίο αναμένουμε καλά αποτελέσματα.

Στο τρίτο κεφάλαιο περιγράφουμε το σύστημα και τις συνθήκες κάτω από τις οποίες
εκτελέστηκαν τα πειράματα, ενώ στο τέταρτο παρουσιάζουμε τα αποτελέσματα. Τέλος
το πέμπτο κεφάλαιο περιέχει μία ανακεφαλαίωση και τα συμπεράσματα της εργασίας
καθώς και μελλοντική δουλειά που μπορεί να γίνει.

Θα γράψουμε για παρόμοιες εργασίες.. Αυτό θα το σπάσουμε σε 2 μέρη:
1) Πως λύνουνε άλλοι γενικά το θέμα του χρονοπρογρ για vm
2) Συγκεκριμένη δουλειά πάνω σε contention aware techniqs
\section{Το πρόβλημα σε βάθος}
Όπως αναφέρθηκε νωρίτερα, ένα από τα μεγαλύτερα προβλήματα που καλείται να
αντιμετωπίσει ένας πάροχος υπηρεσιών είναι αυτό του ανταγωνισμού για κοινούς
πόρους. Όταν κάποιος (πχ μια εταιρεία αποφασίσει να έχει στην ιδιοκτησία του τις
υποδομές στις οποίες θα στηριχθεί, είναι εύκολο να ρυθμίση τη χρήση τους και να
τις μοιράσει στα διάφορα tasks/jobs ανάλογα με τις δικές του ανάγκες. Αυτό
φυσικά γίνεται απλό από τη στιγμή που ο ίδιος ξέρει τί ακριβώς κάνει το κάθε
task, καθώς και τις ανάγκες αυτού. Μπορεί λοιπόν από πριν να αναλύσει τις
δοσοληψίες που θα γίνουν μεταξύ των διεργασιών και δεδομένων των εξαρτήσεων που
ενδεχομένωνς θα παρουσιαστούν μεταξύ τους, να καθορίσει τη σειρά με την οποία
αυτές θα εκτελεστούν ώστε να μειωθεί ο ανταγωνισμός και να αυξηθεί η
συνολική παραγωγικότητα.

Κάτι τέτοιο είναι σαφώς πιο δύσκολο στο περιβάλλον του νέφους και είναι από τους
βασικότερες αιτίες για τις οποίες οι εταιρείες αποφεύγουν αυτή τη λύση. Είναι
υποχρέωση λοιπόν του παρόχου να μπορεί να απομονώσει τις εργασίες που είναι
ιδιαίτερα απαιτητικές και θα επηρεάσουν την απόδοση των υπόλοιπων ώστε να
διασφαλίσει την καλή ποιότητα της υπηρεσίας του. Τα επιμέρους προβλήματα από τα
οποία συνίσταται και το βασικό είναι αρκετά και οι λύσεις τους όχι πάντα
προφανείς.

Στην περίπτωση του IaaS, οι εικονικές μηχανές αντιμετωπίζονται ως μαύρα κουτιά:
δεν είμαστε σε θέση να γνωρίζουμε τί εκτελούν και κατ' επέκταση τις
απαιτήσεις τους σε πόρους. Αυτό έχει ως αποτέλεσμα να μην μπορούμε να
προβλέψουμε από πριν τις επιπτώσεις που θα έχει να δρομολογήσουμε ταυτόχρονα δύο
μηχανές. Αν η επιλογή γίνει με τυχαιοκρατική πολιτική υπάρχει σοβαρή πιθανότητα
οι δύο μηχανές να ανταγωνίζονται είτε για τον δίαυλο προς τη μνήμη είτε για τις
κρυφές μνήμες με αποτέλεσμα η μία να επιβραδύνει σε μεγάλο βαθμό την άλλη. <<
ΕΔΩ ΜΠΟΡΟΥΜΕ / ΠΡΕΠΕΙ ΝΑ ΒΑΛΟΥΜΕ ΚΑΙ ΑΝΤΙΣΤΟΙΧΕΣ ΜΕΤΡΗΣΕΙΣ;;; θα μπορούσαμε να
γεμίσουμε μισή με μία σελίδα εύκολα + θα είχε την αντιστοιχη στήριξη το κείμενό
μας! >>
<<Να γραψω για τον τρόπο που πρέπει να μετράμε τα counters για να κάνουμε on the
fly classification.....>>

\section{Σχετική έρευνα}
Μπλα μπλα μπλα μπλα μπλα μπλα μπλα μπλα μπλα μπλα μπλα μπλα μπλα μπλα μπλα 
μπλα μπλα μπλα μπλα μπλα μπλα μπλα μπλα μπλα μπλα μπλα μπλα μπλα μπλα μπλα 
μπλα μπλα μπλα μπλα μπλα μπλα μπλα μπλα μπλα μπλα μπλα μπλα μπλα μπλα μπλα 
μπλα μπλα μπλα μπλα μπλα μπλα μπλα μπλα μπλα μπλα μπλα μπλα μπλα μπλα μπλα 
\section{LCA: ένας αποδοτικός χρονοδρομολογητής}
Σαν βασική λύση στο πρόβλημα προτάθηκε ένας αρκετά απλός χρονοδρομολογητής που
είχε υποσχόμενες μετρήσεις σε baremetal περιβάλλον.
