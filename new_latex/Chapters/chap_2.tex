\chapter{Τιτλος κεφαλαίου 2}
\section{Αρχες και λοιπά}
Στο δεύτερο κεφάλαιο αναλύουμε σε μεγαλύτερο βάθος το πρόβλημα διατυπώνοντας
τους παράγοντες που το αποτελούν. Επίσης αναφέρουμε σχετική έρευνα από τη
βιβλιογραφία και αναλύουμε έναν σχετικά απλό αλλά αποδοτικό αλγόριθμο από τον
οποίο αναμένουμε καλά αποτελέσματα.

Στο τρίτο κεφάλαιο περιγράφουμε το σύστημα και τις συνθήκες κάτω από τις οποίες
εκτελέστηκαν τα πειράματα, ενώ στο τέταρτο παρουσιάζουμε τα αποτελέσματα. Τέλος
το πέμπτο κεφάλαιο περιέχει μία ανακεφαλαίωση και τα συμπεράσματα της εργασίας
καθώς και μελλοντική δουλειά που μπορεί να γίνει.

Θα γράψουμε για παρόμοιες εργασίες.. Αυτό θα το σπάσουμε σε 2 μέρη:
1) Πως λύνουνε άλλοι γενικά το θέμα του χρονοπρογρ για vm
2) Συγκεκριμένη δουλειά πάνω σε contention aware techniqs
\section{Το πρόβλημα σε βάθος}
Μπλα μπλα μπλα μπλα μπλα μπλα μπλα μπλα μπλα μπλα μπλα μπλα μπλα μπλα μπλα 
μπλα μπλα μπλα μπλα μπλα μπλα μπλα μπλα μπλα μπλα μπλα μπλα μπλα μπλα μπλα 
μπλα μπλα μπλα μπλα μπλα μπλα μπλα μπλα μπλα μπλα μπλα μπλα μπλα μπλα μπλα 
μπλα μπλα μπλα μπλα μπλα μπλα μπλα μπλα μπλα μπλα μπλα μπλα μπλα μπλα μπλα 
\section{Σχετική έρευνα}
Μπλα μπλα μπλα μπλα μπλα μπλα μπλα μπλα μπλα μπλα μπλα μπλα μπλα μπλα μπλα 
μπλα μπλα μπλα μπλα μπλα μπλα μπλα μπλα μπλα μπλα μπλα μπλα μπλα μπλα μπλα 
μπλα μπλα μπλα μπλα μπλα μπλα μπλα μπλα μπλα μπλα μπλα μπλα μπλα μπλα μπλα 
μπλα μπλα μπλα μπλα μπλα μπλα μπλα μπλα μπλα μπλα μπλα μπλα μπλα μπλα μπλα 
\section{LCA: ένας αποδοτικός χρονοδρομολογητής}
Μπλα μπλα μπλα μπλα μπλα μπλα μπλα μπλα μπλα μπλα μπλα μπλα μπλα μπλα μπλα 
μπλα μπλα μπλα μπλα μπλα μπλα μπλα μπλα μπλα μπλα μπλα μπλα μπλα μπλα μπλα 
μπλα μπλα μπλα μπλα μπλα μπλα μπλα μπλα μπλα μπλα μπλα μπλα μπλα μπλα μπλα 
μπλα μπλα μπλα μπλα μπλα μπλα μπλα μπλα μπλα μπλα μπλα μπλα μπλα μπλα μπλα 
